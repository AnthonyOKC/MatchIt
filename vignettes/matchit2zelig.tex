\section{Conducting Analyses after Matching}
\label{sec:analysis}

Any software package may be used for parametric analysis following
\MatchIt.  This includes any of the relevant R packages, or other
statistical software by exporting the resulting matched data sets
using R commands such as {\tt write.csv()} and {\tt write.table()} for
ASCII files or {\tt write.dta()} in the {\tt foreign} package for a
STATA binary file.

When variable numbers of treated and control units have been matched
to each other (e.g., through exact matching, full matching, or k:1
matching with replacement), the weights created by MatchIt should be
used (e.g., in a weighted regression) to ensure that the matched
treated and control groups are weighted up to be similar.  Users
should also remember that the weights created by MatchIt estimate the
average treatment effect on the treated, with the control units
weighted to resemble the treated units.  See below for more detail on
the weights.  With subclassification, estimates should be obtained
within each subclass and then aggregated across subclasses.  When it
is not possible to calculate an effect within each subclass, again the
weights can be used to weight the matched units.

In this section, we show how to use
\hlink{Zelig}{http://gking.harvard.edu/zelig/} with \MatchIt.  Zelig
\citep{ImaKinLau06} is an R package that implements a large variety of
statistical models (using numerous existing R packages) with a single
easy-to-use interface, gives easily interpretable results by
simulating quantities of interest, provides numerical and graphical
summaries, and is easily extensible to include new methods.

\subsection{Quick Overview}

The general syntax is as follows. First, we use \texttt{match.data()}
to create the matched data from the \MatchIt\ output object
(\texttt{m.out}) by excluding unmatched units from the original data,
and including information produced by the particular matching
procedure (i.e., primarily a new data set, but also information that
may result such as weights, subclasses, or the distance measure).
\begin{verbatim}
> m.data <- match.data(m.out)
\end{verbatim}
where {\tt m.data} is the resulting matched data.  Zelig analyses all
use three commands --- \texttt{zelig}, \texttt{setx}, and
\texttt{sim}.  For example, the basic statistical analysis is
performed first:
\begin{verbatim}
> z.out <- zelig(Y ~ treat + x1 + x2, model = mymodel, data = m.data)
\end{verbatim}
where {\tt Y} is the outcome variable, {\tt mymodel} is the selected
model, and {\tt z.out} is the output object from {\tt zelig}.  This
output object includes estimated coefficients, standard errors, and
other typical outputs from your chosen statistical model.  Its
contents can be examined via \texttt{summary(z.out)} or
\texttt{plot(z.out)}, but the idea of Zelig is that these statistical
results are typically only intermediate quantities needed to compute
your ultimate quantities of interest, which in the case of matching
are usually causal inferences.  To get these causal quantities, we use
Zelig's other two commands.  Thus, we can set the explanatory
variables at their means (the default) and change the treatment
variable from a 0 to a 1:
\begin{verbatim}
> x.out <- setx(z.out, treat=0)
> x1.out <- setx(z.out, treat=1)
\end{verbatim}
and finally compute the resulting estimates of the causal effects and
examine a summary:
\begin{verbatim}
> s.out <- sim(z.out, x = x.out, x1 = x1.out)
> summary(s.out)
\end{verbatim}

\subsection{Examples}

We now give four examples using the Lalonde data.  They are meant to
be read sequentially.  You can run these example commands by typing
{\tt demo(analysis)}.  Although we use the linear least squares model
in these examples, a wide range of other models are available in Zelig
(for the list of supported models, see
\hlink{\url{http://gking.harvard.edu/zelig/docs/Models_Zelig_Can.html}}
{http://gking.harvard.edu/zelig/docs/Models_Zelig_Can.html}.

To load the Zelig package after installing it, type
\begin{verbatim}
> library(Zelig)
\end{verbatim}

\begin{description}
\item[Model-Based Estimates] In our first example, we conduct a
  standard parametric analysis and compute quantities of interest in
  the most common way.  We begin with nearest neighbor matching with a
  logistic regression-based propensity score, discarding control units
  outside the convex hull of the treated units
  \citep{KinZen06,KinZen07}:
\begin{verbatim}
> m.out <- matchit(treat ~ age + educ + black + hispan + nodegree + 
                    married + re74 + re75, method = "nearest", discard
                    = "hull.control", data = lalonde)
\end{verbatim}
  Then we check balance using the summary and plot procedures (which
  we don't show here).  When the best balance is achieved, 
  we run the parametric analysis:
\begin{verbatim}
> z.out <- zelig(re78 ~ treat + age + educ + black + hispan + nodegree + 
                  married + re74 + re75, data = match.data(m.out), 
                  model = "ls")
\end{verbatim}
  and then set the explanatory variables at their means (the default)
  and change the treatment variable from a 0 to a 1:
\begin{verbatim}
> x.out <- setx(z.out, treat=0)
> x1.out <- setx(z.out, treat=1)
\end{verbatim}
and finally compute the result and examine a summary:
\begin{verbatim}
> s.out <- sim(z.out, x = x.out, x1 = x1.out)
> summary(s.out)
\end{verbatim}

\item[Average Treatment Effect on the Treated] We illustrate now how
  to estimate the average treatment effect on the treated in a way
  that is quite robust.  We do this by estimating the coefficients in
  the control group alone.

  We begin by conducting nearest neighbor matching with a logistic
  regression-based propensity score:
\begin{verbatim}
> m.out1 <- matchit(treat ~ age + educ + black + hispan + nodegree + 
                    married + re74 + re75, method = "nearest", data = lalonde)
\end{verbatim}
  Then we check balance using the summary and plot procedures (which
  we don't show here).  We reestimate the matching procedure until we
  achieve the best balance possible.  (The running examples here are
  meant merely to illustrate, not to suggest that we've achieved the
  best balance.)  Then we go to Zelig, and in this case choose to fit
  a linear least squares model to the control group only:
\begin{verbatim}
> z.out1 <- zelig(re78 ~ age + educ + black + hispan + nodegree + 
                  married + re74 + re75, data = match.data(m.out1, "control"), 
                  model = "ls")
\end{verbatim}
  where the {\tt "control"} option in {\tt match.data()} extracts only
  the matched control units and {\tt ls} specifies least squares
  regression.  In a smaller data set, this example should probably be
  changed to include all the data in this estimation (using
  \texttt{data = match.data(m.out1)} for the data) and by including
  the treatment indicator (which is excluded in the example since its
  a constant in the control group.)  Next, we use the coefficients
  estimated in this way from the control group, and combine them with
  the values of the covariates set to the values of the treated units.
  We do this by choosing conditional prediction (which means use the
  observed values) in \texttt{setx()}.  The {\tt sim()} command does
  the imputation.
\begin{verbatim}
> x.out1 <- setx(z.out1, data = match.data(m.out1, "treat"), cond = TRUE)
> s.out1 <- sim(z.out1, x = x.out1)
\end{verbatim}
Finally, we obtain a summary of the results by 
\begin{verbatim}
> summary(s.out1)
\end{verbatim}

\item[Average Treatment Effect (Overall)] To estimate the average
  treatment effect, we continue with the previous example and fit the
  linear model to the {\it treatment group}:
\begin{verbatim}
> z.out2 <- zelig(re78 ~ age + educ + black + hispan + nodegree + 
                  married + re74 + re75, data = match.data(m.out1, "treat"), 
                  model = "ls")
\end{verbatim}
We then conduct the same simulation procedure in order to impute the
counterfactual outcome for the {\it control group},
\begin{verbatim}
> x.out2 <- setx(z.out2, data = match.data(m.out1, "control"), cond = TRUE)
> s.out2 <- sim(z.out2, x = x.out2)
\end{verbatim}
In this calculation, Zelig is computing the difference between
observed and the expected values.  This means that the treatment
effect for the control units is the effect of control (observed
control outcome minus the imputed outcome under treatment from the
model).  Hence, to combine treatment effects just reverse the signs of
the estimated treatment effect of controls.
\begin{verbatim}
> ate.all <- c(s.out1$qi$att.ev, -s.out2$qi$att.ev)
\end{verbatim}
The point estimate, its standard error, and the $95\%$ confidence
interval is given by
\begin{verbatim}
> mean(ate.all)
> sd(ate.all)
> quantile(ate.all, c(0.025, 0.975))
\end{verbatim}
  
\item[Subclassification] In subclassification, the average treatment
  effect estimates are obtained separately for each subclass, and then
  aggregated for an overall estimate.  Estimating the treatment
  effects separately for each subclass, and then aggregating across
  subclasses, can increase the robustness of the ultimate results
  since the parametric analysis within each subclass requires only
  local rather than global assumptions.  However, fewer observations
  are obviously available within each subclass, and so this option is
  normally chosen for larger data sets.

  We begin this example by conducting subclassification with four
  subclasses,
\begin{verbatim}
> m.out2 <- matchit(treat ~ age + educ + black + hispan + nodegree + 
                    married + re74 + re75, data = lalonde, 
                    method = "subclass", subclass = 4)
\end{verbatim}
  When balance is as good as we can get it, we then fit a linear
  regression within each subclass by controlling for the estimated
  propensity score (called \texttt{distance}) and other covariates.
  In most software, this would involve running four separate
  regressions and then combining the results.  In Zelig, however, all
  we need to do is to use the {\tt by} option:
\begin{verbatim}
> z.out3 <- zelig(re78 ~ re74 + re75 + distance, 
                  data = match.data(m.out2, "control"), 
                  model = "ls", by = "subclass")
\end{verbatim}
  The same set of commands as in the first example are used to do the
  imputation of the counterfactual outcomes for the treated units:
\begin{verbatim}
> x.out3 <- setx(z.out3, data = match.data(m.out2, "treat"), fn = NULL, 
                 cond = TRUE)
> s.out3 <- sim(z.out3, x = x.out3)
> summary(s.out3)
\end{verbatim}
It is also possible to get the summary result for each subclass. For
example, the following command summarizes the result for the second
subclass.
\begin{verbatim}
> summary(s.out3, subset = 2)
\end{verbatim}
  
\item[How Adjustment After Exact Matching Has No Effect] Regression 
adjustment after exact one-to-one exact matching gives the identical 
answer as a simple, unadjusted difference in means.  General exact 
matching, as implemented in MatchIt, allows one-to-many matches, so to see 
the same result we must weight when adjusting.  In other words: weighted 
regression adjustment after general exact matching gives the identical 
answer as a simple, unadjusted weighted difference in means.  For 
example:

\begin{verbatim}
> m.out <- matchit(treat ~ educ + black + hispan, data = lalonde, 
                   method = "exact")
> m.data <- match.data(m.out)
> ## weighted diff in means 
> weighted.mean(mdata$re78[mdata$treat == 1], mdata$weights[mdata$treat==1]) 
        - weighted.mean(mdata$re78[mdata$treat==0], mdata$weights[mdata$treat==0])
[1] 807
> ## weighted least squares without covariates
> zelig(re78 ~ treat, data = m.data, model = "ls", weights = "weights")

Call:
zelig(formula = re78 ~ treat, model = "ls", data = m.data, weights =
"weights")

Coefficients:
(Intercept)        treat  
       5524          807 

> ## weighted least squares with covariates
> zelig(re78 ~ treat + black + hispan + educ, data = m.data, model = "ls", 
        weights = "weights")

Call:
zelig(formula = re78 ~ treat + black + hispan + educ, model = "ls",
data = m.data, weights = "weights")

Coefficients:
(Intercept)        treat        black       hispan         educ  
        314          807        -1882          258          657  
\end{verbatim}


 \end{description}


%%% Local Variables: 
%%% mode: latex
%%% TeX-master: "matchit"
%%% End: 
