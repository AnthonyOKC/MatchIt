
The main command, \texttt{matchit()}, can be used to implement any of
the matching procedures:
\begin{Schunk}
\begin{Sinput}
> m.out <- matchit(formula, data, method = "nearest", distance = "logit", 
+     distance.options = list(), discard = "none", reestimate = FALSE, 
+     ...)
\end{Sinput}
\end{Schunk}
The command takes some inputs that are common to all matching
procedures and other inputs specific to particular procedures.  The
outputs are standard across all procedures.  We organize the reference
manual in these categories.

\subsubsection{All Matching Methods}
\label{subsubsec:inputs-all}

\begin{enumerate}
  
\item \texttt{formula} takes the usual syntax of R formula, {\tt treat
    \~\ x1 + x2}, where {\tt treat} is a binary treatment indicator
  and {\tt x1} and {\tt x2} are the pre-treatment covariates. Both the
  treatment indicator and pre-treatment covariates must be contained
  in the same data frame, which is specified as {\tt data} (see
  below).  All of the usual R syntax for formulas work here. For
  example, {\tt x1:x2} represents the first order interaction term
  between {\tt x1} and {\tt x2}, and {\tt I(x1 \^\ 2)} represents the
  square term of {\tt x1}. See {\tt help(formula)} for details.
  
\item \texttt{data} specifies the data frame containing the variables
  called in {\tt formula}.  You may find it helpful for the
  diagnostics to specify observation names in the data frame (see
  Section~\ref{rnames}).
  
\item \texttt{method} specifies a matching method. Currently,
  \texttt{exact} (exact matching), \texttt{full} (full matching),
  \texttt{nearest} (nearest neighbor matching), \texttt{optimal}
  (optimal matching), and \texttt{subclass} (subclassification) are
  available. The default is \texttt{nearest}. Note that within each of
  these matching methods, \MatchIt\ offers a variety of options.  See
  Section \ref{methods} for more details.
  
\item \texttt{distance} specifies the method used to estimate the
  distance measure. The default is logistic regression, {\tt logit}.
  Before using any of these techniques, it is best to understand the
  theoretical groundings of these techniques and to evaluate the
  results.  Most of these methods (such as logistic or probit
  regression) are estimating the propensity score, defined as the
  probability of receiving treatment, conditional on the covariates
  (\cite{RosRub83}).  The distance measures used are the predicted
  probabilities from the model (the propensity scores).  Currently,
  the following methods are available:
  \begin{enumerate}
  \item {\tt mahalanobis} computes the Mahalanobis distance measure
    ({\tt mahalanobis()} in the {\tt stats} package).
  \item binomial generalized linear models with various links ({\tt
      glm()} in the {\tt stats} package); \texttt{logit} (logistic
    link), {\tt linear.logit} (logistic link with linear propensity
    score)\footnote{The linear propensity scores are obtained by
      transforming back onto a linear scale}, \texttt{probit} (probit
    link), {\tt linear.probit} (probit link with linear propensity
    score), {\tt cloglog} (complementary log-log link), {\tt
      linear.cloglog} (complementary log-log link with linear
    propensity score), {\tt log} (log link), {\tt linear.log} (log
    link with linear propensity score), {\tt cauchit} (Cauchy CDF
    link), {\tt linear.cauchit} (Cauchy CDF link with linear
    propensity score).

  \item binomial generalized additive model with various links ({\tt
      gam()} in the {\tt mgcv} package); \texttt{GAMlogit} (logistic
    link), {\tt GAMlinear.logit} (logistic link with linear propensity
    score), \texttt{probit} (probit link), {\tt GAMlinear.probit}
    (probit link with linear propensity score), {\tt GAMcloglog}
    (complementary log-log link), {\tt GAMlinear.cloglog}
    (complementary log-log link with linear propensity score), {\tt
      GAMlog} (log link), {\tt GAMlinear.log} (log link with linear
    propensity score), {\tt GAMcauchit} (Cauchy CDF link), {\tt
      GAMlinear.cauchit} (Cauchy CDF link with linear propensity
    score). \citet{HasTib90,BecJac98} and many others discuss the
    generalized additive models.

  \item \texttt{nnet}, neural network model ({\tt nnet()} in the {\tt
      nnet} package).
    \citet{BecKinZen00,Bishop95,KinZen02,White92,Zeng99} among many
    others discuss neural networks.
  
  \item \texttt{rpart}, classification trees ({\tt rpart()} in the
    \texttt{rpart} package). \citet{BreFriOls84,RugKimMar03} and many
    others discuss classification trees.
  \end{enumerate}
  
\item \texttt{distance.options} specifies the optional arguments that
  are passed to the model for estimating the distance measure. The
  input to this argument should be a list.  For example, if the
  distance measure is estimated with a logistic regression, users can
  increase the maximum IWLS iterations by \texttt{distance.options =
    list(maxit = 5000)}.

\item \texttt{discard} specifies whether to discard units that fall
  outside some measure of support of the distance score before
  matching, and not allow them to be used at all in the matching
  procedure.  Note that discarding units may change the quantity of
  interest being estimated.
  \begin{itemize}
  \item \texttt{none} (default) discards no units before matching.
    Use this option when the units to be matched are substantially
    similar, such as in the case of matching treatment and control
    units from a field experiment that was close to (but not fully)
    randomized (e.g., \citealt{Imai05}), when caliper matching will
    restrict the donor pool, or when you do not wish to change the
    quantity of interest and the parametric methods to be used
    post-matching can be trusted to extrapolate.
  \item \texttt{both} discards all units (treated and control) that
    are outside the support of the distance measure. Use this option
    when the units to be matched are substantially different (when
    there is a large degree of non-overlapping support on the distance
    score), such as in the case of measuring the effect of democracy
    on economic growth.
  \item \texttt{control} discards only control units outside the
    support of the distance measure of the treated units.  Use this
    option when the average treatment effect on the treated is of most
    interest and when unwilling to discard non-overlapping treatment
    units (which would change the quantity of interest), such as
    possibly in the case of the effect of job training on those
    individuals that actually participated in a job evaluation program
    or a drug study where interest is in all patients treated with the
    drug.
  \item \texttt{treat} discards only treated units outside the support
    of the distance measure of the control units.  Use this option
    when the average treatment effect on the control units is of most
    interest and when unwilling to discard control units.
  \item \texttt{convex.hull} discards control units not within the
    convex hull of the treated units using the method developed in
    \citep{KinZen05b}.
  \end{itemize}
  
\item \texttt{reestimate} specifies whether the model for distance
  measure should be re-estimated after units are discarded. The input
  must be a logical value. The default is \texttt{FALSE}.
  Re-estimation may be desirable for efficiency reasons, especially if
  many units were discarded and so the post-discard samples are quite
  different from the original samples.

\item \texttt{verbose} specifies whether or not to print out comments
  indicating the status of the matching. The input must be a logical
  value. The default is \texttt{FALSE}.
\end{enumerate}

\subsubsection{Exact Matching}
\label{subsubsec:inputs-exact}

Exact matching is implemented in \MatchIt\ using \texttt{method =
  "exact"}.  Exact matching will be done on all covariates included on
the right-hand side of the \texttt{formula} specified in the \MatchIt\
call.  No \texttt{distance} option is used for exact matching, and
there are no additional options for exact matching.

\subsubsection{Subclassification}
\label{subsubsec:inputs-subclass}

\begin{enumerate}
\item \texttt{subclass} is either (1) a scalar, specifying the number
  of subclasses, or (2) a vector of probabilities bounded between 0
  and 1, to create quantiles of the distance measure using the units
  in the group specified by \texttt{sub.by}.  The default is
  \texttt{subclass = 6}.
\item \texttt{sub.by} specifies by what criteria to subclassify:
  \texttt{"treat"} indicates by the number of treatment units
  (default), \texttt{"control"} indicates by the number of control
  units, and \texttt{"all"} indicates by the total number of units.
\end{enumerate}

\subsubsection{Nearest Neighbor Matching}
\label{subsubsec:inputs-nearest}

\begin{enumerate}
\item \texttt{m.order}  specifies the order in which to match
  treatment units with control units:
  \begin{itemize}
  \item {\tt "largest"} indicates matching from the largest value of
    the distance measure to the smallest. This is the default.
  \item {\tt "smallest"} indicates matching from the smallest value of
    the distance measure to the largest.
  \item {\tt "random"} indicates matching in random order.
  \end{itemize}
\item \texttt{replace} specifies whether each control unit can be
  matched to more than one treated unit.  For matching with
  replacement, \texttt{replace = TRUE}.  If each control is to be used
  as a match at most once (i.e., without replacement), \texttt{replace
    = FALSE}. The default is {\tt FALSE}.
\item \texttt{ratio} specifies the number of control units to match to
  each treated unit, default is {\tt ratio = 1}.  If matching is done
  without replacement and there are fewer control units than ratio
  times the number of eligible treated units (i.e., there are not
  enough control units for the specified method), then the higher
  ratios will have \texttt{NA} in place of the matching unit number in
  \texttt{match.matrix}.
\item \texttt{exact} specifies variables on which to perform exact
  matching within the nearest neighbor matching.  If \texttt{exact} is
  specified, only matches that exactly match on the covariates in
  \texttt{exact} will be allowed.  Within the matches that match on
  the variables in \texttt{exact}, the match with the closest distance
  measure will be chosen.  \texttt{exact} should be entered as a
  vector of variable names (\texttt{exact = c("X1", "X2")}) that are
  names of variables in \texttt{data}.
\item \texttt{caliper} specifies the number of standard deviations of
  the distance measure within which to draw control units, default=0.
  If a caliper is specified, the matches are restricted to being
  within the caliper and a control unit within the caliper for a
  treated unit is randomly selected as the match for that treated
  unit.  If \texttt{caliper != 0}, there are two additional options:
  \begin{itemize} 
  \item \texttt{calclosest} specifies whether to take the nearest
    available match if no matches are available within the
    \texttt{caliper}. The default is {\tt FALSE}.
  \item \texttt{mahvars} specifies variables on which to perform
    Mahalanobis-metric matching within each caliper (default=NULL).
    Variables should be entered as a vector of variable names
    (\texttt{mahvars = c("X1", "X2")}) that are names of variables in
    \texttt{data}.  If \texttt{mahvars} is specified without
    \texttt{caliper}, the caliper is set to 0.25.
  \end{itemize}
\item \texttt{subclass} and \texttt{sub.by}.  See Section
  \ref{subsubsec:subclass} for more details on these options.  If a
  \texttt{subclass} is specified within \texttt{method = "nearest"},
  the matched units will be placed into subclasses after the nearest
  neighbor matching is completed.
\end{enumerate}

\subsubsection{Optimal Matching}
\label{subsubsec:inputs-optimal}

The available options are listed below.
\begin{enumerate}
\item {\tt ratio} specifies the number of control units to be matched
  to each treatment unit, the default is {\tt ratio = 1}.
\item {\tt ...} represents additional inputs that can be passed to the
  {\tt fullmatch()} function in the {\tt optmatch} package. See {\tt
    help(fullmatch)} or
  \hlink{http://www.stat.lsa.umich.edu/\~{}bbh/optmatch.html}{http://www.stat.lsa.umich.edu/\~bbh/optmatch.html}
  for details.
\end{enumerate}

\subsubsection{Full Matching}
\label{subsubsec:inputs-full}

\begin{enumerate}
\item {\tt ...} represents additional inputs that can be passed to the
  {\tt fullmatch()} function in the {\tt optmatch} package. See {\tt
    help(fullmatch)} or
  \hlink{http://www.stat.lsa.umich.edu/\~{}bbh/optmatch.html}{http://www.stat.lsa.umich.edu/\~bbh/optmatch.html}
  for details.
\end{enumerate}

\subsubsection{Genetic Matching}
\label{subsubsec:inputs-genetic}

The available options are listed below.
\begin{enumerate}
\item {\tt ratio} specifies the number of control units to be matched
  to each treatment unit, the default is {\tt ratio = 1}. This option
  does not seem work well as it is implemented in {\tt Matching}
  package.
\item {\tt ...} represents additional inputs that can be passed to the
  {\tt GenMatch()} function in the {\tt Matching} package. See {\tt
    help(GenMatch)} or
  \hlink{http://sekhon.polisci.berkeley.edu/library/Matching/html/GenMatch.html}{http://sekhon.polisci.berkeley.edu/library/Matching/html/GenMatch.html}
  for details.
\end{enumerate}
