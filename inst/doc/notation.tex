\documentclass[oneside,letterpaper,titlepage,12pt]{article}
%\usepackage[ae,hyper]{/usr/lib/R/share/texmf/Rd}
\usepackage{makeidx}
\usepackage{graphicx}
\usepackage{natbib}
\usepackage[reqno]{amsmath}
\usepackage{amssymb}
\usepackage{verbatim}
\usepackage{epsf}
\usepackage{url}
\usepackage{html}
\usepackage{dcolumn}
\usepackage{longtable}
\usepackage{vmargin}
\setpapersize{USletter}
\newcolumntype{.}{D{.}{.}{-1}}
\newcolumntype{d}[1]{D{.}{.}{#1}}
%\pagestyle{myheadings}
\htmladdtonavigation{
  \htmladdnormallink{%
    \htmladdimg{http://gking.harvard.edu/pics/home.gif}}
  {http://gking.harvard.edu/}}
\newcommand{\hlink}{\htmladdnormallink}

\bodytext{ BACKGROUND="http://gking.harvard.edu/pics/temple.setcounter"}
\setcounter{tocdepth}{3}

\parindent=0cm
\newcommand{\MatchIt}{\textsc{MatchIt}}

\begin{document}

\begin{center}
Notation for \MatchIt \\
Elizabeth Stuart \\
\end{center}

This document details the notation to be used in the matching paper to accompany \MatchIt,
as discussed in conference call on April 27, 2004. \\


First define notation for individual $i$:
\begin{itemize}
\item $Y_{1i}=$ individual $i$'s potential outcome under treatment 
\item $Y_{0i}=$ individual $i$'s potential outcome under control
\item $T_i=$ individual $i$'s observed treatment assignment 
\begin{itemize} \item $T_i=1$ means individual $i$ receives treatment
		\item $T_i=0$ means individual $i$ receives control
\end{itemize}
\item $Y_{i0}$ and $Y_{i1}$ are considered fixed quantities.  Which one is observed depends on the random variable $T_i$.
\item $y_i=T_i Y_{1i} + (1-T_i) Y_{0i}$ is individual $i$'s observed outcome
\item Let $y_{1i}=(Y_{1i}|T_i=1)$.  Similarly, $y_{0i}=(Y_{0i}|T_i=0)$.  One of these is observed for individual $i$.
\item Let $\tilde{y}_{0i}$ be a draw from the posterior distribution of $Y_{0i}$, given that $T_i=1$: $\tilde{y}_{0i} \sim P(Y_{0i}|T_i=1)$. 
Similarly, let  $\tilde{y}_{1i}$ be a draw from the posterior distribution of $Y_{1i}$, 
given that $T_i=0$: $\tilde{y}_{1i} \sim P(Y_{1i}|T_i=0)$.
[I'm not crazy about this 
notation...is the ``$|T_i=1$'' necessary, or could we just use $\tilde{y}_{0i} \sim P(Y_{0i})$?  
We can still have $\tilde{y}_{0i}$ even if $T_i=0$, we just won't ever actually use it.  But it does still exist in some
sense.].  
\item Consider variables $X_i$.  If $X_{0i}=X_{1i}=X_i$ then $X$ is a ``proper covariate'' in that it is not affected by treatment assignment.  Only proper
covariates should be used in matching procedure.  This is always an assumption--we never observe $X_{0i}$ and $X_{1i}$ for individual $i$.
\item We think of the capital $Y$'s as corresponding to ``true'' values ($Y_{0i}$, $Y_{1i}$) while the lower-case $y$'s represent realized values:
lower-case $y$'s without a tilde are observed values; lower-case $y$'s with a tilde are draws from the posterior distribution.
\end{itemize}

For individual $i$, the potentially observed values can be represented by the following 2x2 table:

\begin{center}
\begin{tabular}{c|c|c|}
\multicolumn{3}{c}{\hspace*{.8cm} $T_i=0$ \hfill $T_i=1$} \\
\cline{2-3}
$Y_{0i}$ & $y_{0i}$, $X_i$ & $\tilde{y}_{0i}$, $X_i$ \\
\cline{2-3}
$Y_{1i}$ & $\tilde{y}_{1i}$, $X_i$ & $y_{1i}$, $X_i$ \\
\cline{2-3}
\end{tabular}
\end{center}
 
For each individual, we are in either Column 1 or Column 2 (depending on treatment assignment).  Within each column, 1 number will be observed and 1 will be a drawn
value from the posterior distribution (i.e., the true value for that cell is missing). \\


Now consider $n$ individuals observed, with $n_1$ in the treated group and $n_0$ in the control group ($n=n_0+n_1$).  Then we have the following notation:
\begin{itemize}
\item $n_1=\sum_{i=1}^n T_i$, $n_0=\sum_{i=1}^n (1-T_i)$
\item ${\bf y_0} = \{y_{0i}\} = \{y_{i}|T_i=0\}$.  ${\bf y_0}$ is of length $n_0$.
\item ${\bf y_1} = \{y_{1i}\} = \{y_{i}|T_i=1\}$.  ${\bf y_1}$ is of length $n_1$.
\item ${\bf y} = \{ {\bf y_0}, {\bf y_1}\}$.  ${\bf y}$ is of length $n$.
\item $\overline{y}_1 = \frac{\sum_{i=1}^n T_i Y_{1i}}{\sum_{i=1}^n T_i}=\frac{\sum_{i=1}^{n_1} y_{1i}}{n_1}$
\item $\overline{y}_0 = \frac{\sum_{i=1}^n (1-T_i) Y_{0i}}{\sum_{i=1}^n (1-T_i)}=\frac{\sum_{i=1}^{n_0} y_{0i}}{n_0}$
\item Observed sample variances would be calculated in a similar way (using ${\bf y_0}$ and ${\bf y_1}$).
\end{itemize}

One point to make sure we mention in the write-up is to say what OLS assumes.  OLS with a treatment indicator does:
$$E(y_i) = \alpha + \beta_0 T_i + {\boldsymbol \beta} {\bf X_i},$$
which assumes that the models of the potential outcomes follow parallel lines:
$$E(y_{1i}) = \alpha + \beta_0 + {\boldsymbol \beta} {\bf X_i},$$
$$E(y_{0i}) = \alpha + {\boldsymbol \beta} {\bf X_i}.$$
We can also have a graphical representation of this, perhaps with some examples of outcomes that clearly don't have that nice parallel linear relationship,
and show that this would be a very special case of what we're advocating.

\end{document}