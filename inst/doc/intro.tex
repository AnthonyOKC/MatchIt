
\section{What \MatchIt\ Does}

\MatchIt\ implements the suggestions of \citet*{HoImaKin06} for
improving parametric statistical models and reducing model dependence
by preprocessing data with semi-parametric and non-parametric matching
methods.  After preprocessing with \MatchIt, researchers can use
whatever parametric model and software they would have used without
\MatchIt, without other modification, and produce inferences that are
substantially more robust and less sensitive to modeling assumptions.
(In addition, you may wish to use
\hlink{Zelig}{http://gking.harvard.edu/zelig/} \citep{ImaKinLau04} for
subsequent parametric analyses, as it is designed for maximum
convenience in analyzing \MatchIt\ datasets.)  \MatchIt\ reduces the
dependence of causal inferences on commonly made, but hard-to-justify,
statistical modeling assumptions via the largest range of
sophisticated matching methods of any software we know of.  The
program includes most existing approaches to matching and even enables
users to access methods implemented in other programs through its
single, unified, and easy-to-use interface.  (In addition, we have
written \MatchIt\ so that adding new matching methods to the software
is as easy for anyone with the inclination as it is for us.)

\section{Software Requirements} 
\label{sec:require}

\MatchIt\ works in conjunction with the R programming language and
statistical software, and will run on any platform where R is
installed (Windows, Unix, or Mac OS X).  R is available free for
download at the Comprehensive R Archive Network (CRAN) at
\hlink{http://cran.r-project.org/}{http://cran.r-project.org/}.
\MatchIt\ has been tested on the most recent version of R.  A good way
to learn R, if you don't know it already, is to learn Zelig (available
at
\hlink{http://gking.harvard.edu/zelig}{http://gking.harvard.edu/zelig})
which includes a self-contained introduction to R and can be used to
analyze the matched data after running \MatchIt.

\section{Installing \MatchIt}
\label{sec:install}

To install \MatchIt\ for all platforms, type at the R command prompt,
\begin{verbatim}
> install.packages("MatchIt")
\end{verbatim}
and \MatchIt\ will install itself onto your system automatically.
(During the installation process you may either decide to keep or
discard the installation files, which will not affect the way
\MatchIt\ runs.)  

\section{Loading \MatchIt} \label{sec:load}

You need to install \MatchIt\ only once, but you must load it prior to
each use.  You can do this at the R prompt:
\begin{verbatim}
> library(MatchIt)
\end{verbatim}

Alternatively, you can specify R to load \MatchIt\ automatically at
launch by editing the {\tt Rprofile} file located in the R program
subdirectory, e.g.  \texttt{C:/R/rw2011/etc/}, for Windows systems or
the {\tt .Rprofile} file located in the home directory for Unix/Linux
and Mac OS X systems, and adding this line:
\begin{verbatim}
options(defaultPackages = c(getOption("defaultPackages"), "MatchIt"))
\end{verbatim}
For this change to take effect, you need to restart R.

\section{Updating \MatchIt}

We recommend that you periodically update \MatchIt\ at the R prompt by typing:
\begin{verbatim}
> update.packages()
> library(MatchIt)
\end{verbatim}
which will update all the libraries including \MatchIt\ and load the
new version of \MatchIt.


%%% Local Variables: 
%%% mode: pdflatex
%%% TeX-master: "matchit"
%%% End: 
