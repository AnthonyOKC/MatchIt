\section{Conducting Analyses after Matching}
\label{sec:analysis}

Although any program may be used for parametric analysis following
\MatchIt, this section shows how to use
\hlink{Zelig}{http://gking.harvard.edu/zelig/} with \MatchIt.  (The
resulting matched data sets can also be exported to other statistical
programs using commands such as {\tt write.csv()} and {\tt
  write.table()} for ASCII files, and {\tt write.dta()} in {\tt
  foreign} package for a STATA binary file.)  Zelig
\citep{ImaKinLau04} is an R package that implements a large variety of
statistical models with a single easy-to-use interface, gives easily
interpretable results by simulating quantities of interest, provides
numerical and graphical summaries, and is easily extensible.

\subsection{Quick Overview}

The general syntax is as follows. First, we use \texttt{match.data()}
to create the matched data from the \MatchIt\ output object
(\texttt{m.out}) by excluding unmatched units from the original data,
and including information about the particular matching procedure
(i.e., weights, subclasses, and the distance measure).
\begin{verbatim}
> m.data <- match.data(m.out)
\end{verbatim}
where {\tt m.data} is the resulting matched data.  Next, we analyze
the matched data set
\begin{verbatim}
> z.out <- zelig(Y ~ treat + x1 + x2, model = mymodel, data = m.data)
\end{verbatim}
where {\tt Y} is the outcome variable, {\tt mymodel} is the selected
model, and {\tt z.out} is the output object from {\tt zelig}.

\subsection{Examples}

We now give four examples using the Lalonde data.  They are meant to
be read sequentially.  You can run these example commands by typing
{\tt demo(analysis)}.  Although we use the linear least squares model
in these examples, a wide range of other models are available in Zelig
(for the list of supported models, see
\hlink{http://gking.harvard.edu/zelig/docs/Models\_Zelig\_Can.html}{http://gking.harvard.edu/zelig/docs/Models_Zelig_Can.html}.

To load the Zelig package after installing it, type
\begin{verbatim}
> library(Zelig)
\end{verbatim}

\begin{description}
\item[Model-Based Estimates] In our first example, we conduct a
  standard parametric analysis and compute quantities of interest in
  the most common way.  We begin with nearest neighbor matching with a
  logistic regression-based propensity score, discarding with the
  convex.hull option:
\begin{verbatim}
> m.out0 <- matchit(treat ~ age + educ + black + hispan + nodegree + 
                    married + re74 + re75, method = "nearest", discard
                    = "convex.hull", data = lalonde)
\end{verbatim}
  Then we check balance using the summary and plot procedures (which
  we don't show here).  When we get balance as good as possible, we
  run the parametric analysis:
\begin{verbatim}
> z.out0 <- zelig(re78 ~ age + educ + black + hispan + nodegree + 
                  married + re74 + re75, data = match.data(m.out1), 
                  model = "ls")
\end{verbatim}
  and then set the explanatory variables at their means (the default)
  and change the treatment variable from a 0 to a 1:
\begin{verbatim}
> x.out0 <- setx(z.out0, treat=0)
> x1.out0 <- setx(z.out0, treat=1)
\end{verbatim}
and finally compute the result and examine a summary:
\begin{verbatim}
> s.out0 <- sim(z.out1, x = x.out1)
> summary(s.out0)
\end{verbatim}

\item[Average Treatment Effect on the Treated] We illustrate now how
  to estimate the average treatment effect on the treated in a way
  that is quite robust.  We do this by estimating the coefficients in
  the control group alone.

  We begin by conducting nearest neighbor matching with a logistic
  regression-based propensity score:
\begin{verbatim}
> m.out1 <- matchit(treat ~ age + educ + black + hispan + nodegree + 
                    married + re74 + re75, method = "nearest", data = lalonde)
\end{verbatim}
  Then we check balance using the summary and plot procedures (which
  we don't show here).  We reestimate the matching procedure until we
  get balance as good as possible.  Then we go to Zelig, and in this
  case choose to fit a linear least squares model to the control group
  only:
\begin{verbatim}
> z.out1 <- zelig(re78 ~ age + educ + black + hispan + nodegree + 
                  married + re74 + re75, data = match.data(m.out1, "control"), 
                  model = "ls")
\end{verbatim}
  where the {\tt "control"} option in {\tt match.data()} extracts only
  the matched control units and {\tt ls} specifies least squares
  regression.  In a smaller data set, this example should probably be
  changed to include all the data in this estimation (using
  \texttt{data = match.data(m.out1)} for the data) and by including
  the treatment indicator (which is excluded in the example since its
  a constant in the control group.)  Next, we use the coefficients
  estimated in this way from the control group, and combine them with
  the values of the covariates set to the values of the treated units.
  We do this by choosing conditional prediction (which means use the
  observed values) in \texttt{setx()}.  The {\tt sim()} command does
  the imputation.
\begin{verbatim}
> x.out1 <- setx(z.out1, data = match.data(m.out1, "treat"), cond = TRUE)
> s.out1 <- sim(z.out1, x = x.out1)
\end{verbatim}
Finally, we obtain a summary of the results by 
\begin{verbatim}
> summary(s.out1)
\end{verbatim}

\item[Average Treatment Effect (Overall)] To estimate the average
  treatment effect, we continue with the previous example and fit the
  linear model to the {\it treatment group}:
\begin{verbatim}
> z.out2 <- zelig(re78 ~ age + educ + black + hispan + nodegree + 
                  married + re74 + re75, data = match.data(m.out1, "treat"), 
                  model = "ls")
\end{verbatim}
We then conduct the same simulation procedure in order to impute the
counterfactual outcome for the {\it control group},
\begin{verbatim}
> x.out2 <- setx(z.out2, data = match.data(m.out1, "control"), cond = TRUE)
> s.out2 <- sim(z.out2, x = x.out2)
\end{verbatim}
In this calculation, Zelig is computing the difference between
observed and the expected values.  This means that the treatment
effect for the control units is the effect of control (observed
control outcome minus the imputed outcome under treatment from the
model).  Hence, to combine treatment effects just reverse the signs of
the estimated treatment effect of controls.
\begin{verbatim}
> ate.all <- c(s.out1$qi$ate.ev, -s.out2$qi$ate.ev)
\end{verbatim}
The point estimate, its standard error, and the $95\%$ confidence
interval is given by
\begin{verbatim}
> mean(ate.all)
> sd(ate.all)
> quantile(ate.all, c(0.025, 0.975))
\end{verbatim}
  
\item[Subclassification] In subclassification, the average treatment
  effect estimates are obtained separately for each subclass, and then
  aggregated for an overall estimate.  Estimating the treatment
  effects separately for each subclass, and then aggregating across
  subclasses, can increase the robustness of the ultimate results
  since the parametric analysis within each subclass requires only
  local rather than global assumptions.  However, fewer observations
  are obviously available within each subclass, and so this option is
  normally chosen for larger data sets.

  We begin this example by conducting subclassification with four
  subclasses,
\begin{verbatim}
> m.out2 <- matchit(treat ~ age + educ + black + hispan + nodegree + 
                    married + re74 + re75, data = lalonde, 
                    method = "subclass", subclass = 4)
\end{verbatim}
  When balance is as good as we can get it, we then fit a linear
  regression within each subclass by controlling for the estimated
  propensity score (called \texttt{distance}) and other covariates.
  In most software, this would involve running four separate
  regressions and then combining the results.  In Zelig, however, all
  we need to do is to use the {\tt by} option:
\begin{verbatim}
> z.out3 <- zelig(re78 ~ re74 + re75 + distance, 
                  data = match.data(m.out2, "control"), 
                  model = "ls", by = "subclass")
\end{verbatim}
  The same set of commands as in the first example are used to do the
  imputation of the counterfactual outcomes for the treated units:
\begin{verbatim}
> x.out3 <- setx(z.out3, data = match.data(m.out2, "treat"), fn = NULL, 
                 cond = TRUE)
> s.out3 <- sim(z.out3, x = x.out3)
> summary(s.out3)
\end{verbatim}
It is also possible to get the summary result for each subclass. For
example, the following command summarizes the result for the second
subclass.
\begin{verbatim}
> summary(s.out3, subset = 2)
\end{verbatim}
  
\end{description}


%%% Local Variables: 
%%% mode: latex
%%% TeX-master: "matchit"
%%% End: 
