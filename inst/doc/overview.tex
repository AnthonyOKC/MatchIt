
\MatchIt\ is designed for causal inference with a dichotomous treatment
variable and a set of pretreatment control variables.  Any number or
type of dependent variables can be used.  (If you are interested in
the causal effect of more than one variable in your data set, run
\MatchIt\ separately for each one; it is unlikely in any event that
any one parametric model will produce valid causal inferences for more
than one treatment variable at a time.)  \MatchIt\ can be used for
other types of causal variables by dichotomizing them, perhaps in
multiple ways \citep[see also][]{ImaDyk04}.  \MatchIt\ works for
experimental data, but is usually used for observational studies where
the treatment variable is not randomly assigned by the investigator,
or the random assignment goes awry.

We adopt the same notation as in \citet*{HoImaKin07}.  Unless
otherwise noted, let $i$ index the $n$ units in the data set, $n_1$
denote the number of treated units, $n_0$ denote the number of control
units (such that $n=n_0+n_1$), and $x_i$ indicate a vector of
pretreatment (or control) variables for unit $i$.  Let $t_i=1$ when
unit $i$ is assigned treatment, and $t_i=0$ when unit $i$ is assigned
control.  (The labels ``treatment'' and ``control'' and values 1 and 0
respectively are arbitrary and can be switched for convenience, except
that some methods of matching are keyed to the definition of the
treated group.)  Denote $y_i(1)$ as the potential outcome of unit $i$
under treatment --- the value the outcome variable would take if $t_i$
were equal to 1, whether or not $t_i$ in fact is 0 or 1 -- and
$y_i(0)$ the potential outcome of unit $i$ under control --- the value
the outcome variable would take if $t_i$ were equal to 0, regardless
of its value in fact.  The variables $y_i(1)$ and $y_i(0)$ are jointly
unobservable, and for each $i$, we observe one
$y_i=t_iy_i(1)+(1-t_i)y_i(0)$, and not the other.  

Also denote a fixed vector of exogneous, pretreatment measured
confounders as $X_i$.  These variables are defined in the hope or
under the assumption that conditioning on them appropriately will make
inferences ignorable.  Measures of balance should be computed with
respect to all of $X$, even if some methods of matching only use some
components.

\section{Preprocessing via Matching}

If $t_i$ and $X_i$ were independent, we would not need to control for
$X_i$, and any parametric analysis would effectively reduce to a
difference in means of $Y$ for the treated and control groups.  The
goal of matching is to preprocess the data prior to the parametric
analysis so that the actual relationship between $t_i$ and $X_i$ is
eliminated or reduced without introducing bias and or increasing
inefficiency too much.  

When matching we select, duplicate, or selectively drop observations
from our data, and we do so without inducing bias as long as we use a
rule that is a function only of $t_i$ and $X_i$ and does not depend on
the outcome variable $Y_i$.  Many methods that offer this
preprocessing are included here, including exact, subclassification,
nearest neighbor (including propensity score), optimal, and genetic
matching.  In order to avoid changing the quantity of interest, most
\MatchIt\ routines work by retaining all treated units and only
choosing control units to include in the final data set; this enables
one to estimate the average treated effect on the treated (the purpose
of which is described in Section \ref{s:qoi}).

\MatchIt\ implements and evaluates the choice of the rules for
matching.  Matching sometimes increases efficiency by eliminating
heterogeneity or deleting observations outside of an area where a
model can reasonably be used to extrapolate, but one needs to be
careful not to lose too many observations in matching or efficiency
will drop more than the reduction in bias that is achieved.

The simplest way to obtain good matches (as defined above) is to use
one-to-one exact matching, which pairs each treated unit with one
control unit for which the values of $X_i$ are identical.  However,
with many covariates and finite numbers of potential matches, 
sufficient exact matches often cannot be found.  Indeed, many of the
other methods implemented in \MatchIt\ attempt to balance the overall
covariate distributions as much as possible, when sufficient
one-to-one exact matches are not available.

A key point in \citet*{HoImaKin07} is that matching methods by
themselves are not methods of estimation: Every use of matching in the
literature involves an analysis step following the matching procedure,
but almost all analyses use a simple difference in means.  This
procedure is appropriate only if exact matching was conducted.  In
almost all other cases, some adjustment is required, and there is no
reason to degrade your inferences by using an inferior method of
analysis such as a difference in means even when improving your
inferences via preprocessing.  Thus, with \MatchIt, you can improve
your analyses in two ways.  In fact, \MatchIt\ analyses are ``doubly
robust'' in that if \emph{either} the matching analysis \emph{or} the
analysis model is correct (but not necessarily both) your inferences
will be statistically consistent.  In practice, the modeling choices
you make at the analysis stage will be much less consequential if you
match first.

\section{Checking Balance}
\label{sec:balance-sum}

The goal of matching is to create a data set that looks closer to one
that would result from a perfectly blocked (and possibly randomized)
experiment.  When we get close, we break the link between the treatment
variable and the pretreatment controls, which makes the parametric
form of the analysis model less relevant or irrelevant entirely.  To
break this link, we need the distribution of covariates to be the same
within the matched treated and control groups.

A crucial part of any matching procedure is, therefore, to assess how
close the (empirical) covariate distributions are in the two groups,
which is known as ``balance.''  Because the outcome variable is not
used in the matching procedure, any number of matching methods can be
tried and evaluated, and the one matching procedure that leads to the
best balance can be chosen.  \MatchIt\ provides a number of ways to
assess the balance of covariates after matching, including numerical
summaries such as the ``mean Diff.'' (difference in means) or the
difference in means divided by the treated group standard deviation,
and summaries based on quantile-quantile plots that compare the
empirical distributions of each covariate.  The widely used procedure
of doing t-tests of the difference in means is highly misleading and
should never be used to assess balance; see \citet{ImaKinStu07}.

These balance diagnostics should be performed on all variables in $X$,
even if some are excluded from one of the matching procedures.

\section{Conducting Analyses after Matching}\label{s:qoi}

The most common way that parametric analyses are used to compute
quantities of interest (without matching) is by (statistically)
holding constant some explanatory variables, changing others, and
computing predicted or expected values and taking the difference or
ratio, all by using the parametric functional form.  In the case of
causal inference, this would mean looking at the effect on the
expected value of the outcome variable when changing $T$ from 0 to 1,
while holding constant the pretreatment control variables $X$ at their
means or medians.  This, and indeed any other appropriate analysis
procedure, would be a perfectly reasonable way to proceed with
analysis after matching.  If it is the chosen way to proceed, then
either treated or control units may be deleted during the matching
stage, since the same parametric structure is assumed to apply to all
observations.

In other instances, researchers wish to reduce the assumptions
inherent in their statistical model and so want to allow for the
possibility that their treatment effect to vary over observations.  In
this situation, one popular quantity of interest used is the
\emph{average treatment effect on the treated} (ATT).  For example,
for the treated group, the potential outcomes under control, $Y_i(0)$,
are missing, whereas the outcomes under treatment, $Y_i(1)$, are
observed, and the goal of the analysis is to impute the missing
outcomes, $Y_i(0)$ in observations where $T_i=1$.  We do this via
simulation using a parametric statistical model such as regression,
logit, or others (as described below).  Once those potential outcomes
are imputed from the model, the estimate of individual $i$'s treatment
effect is $Y_i(1)-\widehat{Y}_i(0)$ where $\widehat{Y}_i(0)$ is a
predicted value of the dependent variable for unit $i$ under the
counterfactual condition where $T_i=0$.  The in-sample average
treatment effect for the treated individuals can then be obtained by
averaging this difference over all observations $i$ where in fact
$T_i=1$.  Most \MatchIt\ algorithms retain all treated units, and
choose some subset of or repeated units from the control group, so
that estimating the ATT is straightforward.  If one chooses options
that allow matching with replacement, or any solution that has
different numbers of controls (or treateds) within each strata, then
then the parametric analysis following matching must accomodate these
procedures, such as using fixed effects or weights, as appropriate.
(Similar procedures can also be used to estimate various other
quantities of interest such as the average treatment effect by
computing it for all observations, but then one must be aware that the
quantity of interest may change during the matching procedure.)

The imputation from the model can be done in at least two ways.
Recall that the model is used to impute \emph{the value that the
  outcome variable would take among the treated units if those treated
  units were actually controls}.  Thus, one reasonable approach would
be to fit a model to the matched data and create simulated predicted
values of the dependent variable for the treated units with $T_i$
switched counterfactually from 1 to 0.  An alternative approach would
be to fit a model without $T$ by using only the outcomes of the
matched control units (i.e., using only observations where $T_i=0$).
Then, given this fitted model, the missing outcomes $Y_i(0)$ are
imputed for the matched treated units by using the values of the
explanatory variables for the treated units.  The first approach will
usually have lower variance, since all observations are used, and the
second may have less bias, since no assumption of constant parameters
is needed.  See \citet*{HoImaKin07} for more details.

Other quantities of interest can also be computed at the parametric
stage, following any procedures you would have followed in the absence
of matching.  The advantage is that if matching is done well your
answers will be more robust to many small changes in parametric
specification.


%%% Local Variables: 
%%% mode: latex
%%% TeX-master: "matchit"
%%% End: 
