\section{\texttt{plot()}: Graphical Summaries of Balance}

\subsection{Description}
The \texttt{plot()} command allows you to check the distributions of
covariates in the assignment model, squares, and interactions, and
within each subclasses if specified.  

\subsection{Syntax}

\begin{verbatim}
> plot(m.out, discrete.cutoff = 5, type = "QQ", 
       numdraws = 5000, interactive = TRUE, which.xs = NULL, ...)
\end{verbatim}

\subsection{Arguments}

\begin{itemize}
\item {\tt type}: type of output graph. \texttt{type = "QQ"}
  (default) outputs empirical quantile-quantile plots of each
  covariate to check balance of marginal distributions. Alternatively,
  \texttt{type = "jitter"} outputs jitter plots of the propensity
  score for treated and control units.
  
\item {\tt discrete.cutoff}: For quantile-quantile plots, discrete
  covariates that take 5 or fewer values are jittered for visibility.
  This may be changed by setting this argument to any other positive
  integer.
  
\item {\tt interactive}: If \texttt{TRUE} (default), users can
  identify individual units by clicking on the graph with the left
  mouse button, and (when applicable) choose subclasses to plot.
  
\item {\tt which.xs}: Specifies particular covariate names in a
  character vector to plot only a subset of the covariates.

\item {\tt subclass}: If \texttt{interactive = FALSE}, users can
  specify which subclass to plot. 

\end{itemize}

\subsection{Output Values}

\begin{itemize}
\item Empirical quantile-quantile plot: This graph plots covariate
  values that fall in (approximately) the same quantile of treated and
  control distributions.  Control unit quantile values are plotted on
  the x-axis, and treated unit quantile values are plotted on the
  y-axis.  If values fall below the 45 degree line, control units
  generally take lower values of the covariate.  Data points that fall
  exactly on the 45 degree line indicate that the marginal
  distributions are identical.
  
\item Jitter plots: This graph plots jittered estimated propensity
  scores of treated and control units.  Dark diamonds indicate matched
  units and grey diamonds indicate unmatched or discarded units.  The
  area of the diamond is proportional to the weights. Vertical lines
  are plotted if subclassification is used.

\end{itemize}


%%% Local Variables: 
%%% mode: latex
%%% TeX-master: "matchit"
%%% End: 

