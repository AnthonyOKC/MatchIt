\section{\texttt{plot()}: Graphical Summaries of Balance Diagnostics}

The \texttt{plot()} command allows you to check the distributions of
covariates in the assignment model, squares, and interactions, and
within each subclasses if specified.  The graphs present:
\begin{enumerate}
\item Q-Q plots of each covariate to check balance of marginal
  distributions (\texttt{type = "QQ"} (default)).  This graph plots
  covariate values that fall in (approximately) the same quantile of
  treated and control distributions.  Control unit quantile values are
  plotted on the x-axis, and treated unit quantile values are plotted
  on the y-axis.  If values fall below the 45 degree line, control
  units generally take lower values of the covariate.  Data points
  that fall exactly on the 45 degree line indicate that the marginal
  distributions are identical.  Discrete covariates that take 5 or
  fewer values are jittered for visibility.  This may be changed by
  setting the option \texttt{discrete.cutoff}.  

\item Since there can be many covariates (and many subclasses in
  subclassification), the default Q-Q plot is interactive.  That is,
  after plotting a $3 \times 2$ panel of raw and matched Q-Q plots for
  three covariates, user input is required to plot the next panel, if
  necessary.  Similarly, user input is required when plotting by
  subclass.  This interactivity can be turned off by setting the input
  \texttt{interactive = F}.  When subclassifying, users can then
  specify the subclass by the \texttt{subclass} input.  Users can also
  specify specific covariates to plot by entering names of covariates
  in \texttt{which.xs}.

\item Jitter plots of the propensity score for treated and control
  units (\texttt{type = "jitter"}).  If \texttt{interactive=TRUE} (the
  default), the user can identify individual units by clicking on the
  graph with the left mouse button.
\end{enumerate}

