\section{\texttt{plot()}: Graphical Summaries of Balance}


\subsection{Plot options for the matchit object}
The \texttt{plot()} command allows you to check the distributions of
propensity scores and covariates in the assignment model, squares, and
interactions, and within each subclasses if specified.

\subsubsection{Syntax}

\begin{verbatim}
> plot(m.out, discrete.cutoff = 5, type = "QQ", 
       numdraws = 5000, interactive = TRUE, which.xs = NULL, ...)
\end{verbatim}

\subsubsection{Arguments}

\begin{itemize}
\item {\tt type}: type of output graph. \texttt{type = "QQ"}
  (default) outputs empirical quantile-quantile plots of each
  covariate to check balance of marginal distributions. Alternatively,
  \texttt{type = "jitter"} outputs jitter plots of the propensity
  score for treated and control units.  Finally, \texttt{type="hist"} outputs  
  histograms of the propensity score in the original treated and control groups
  and weighted histograms of the propensity score in the matched treated and control groups.
  
\item {\tt discrete.cutoff}: For quantile-quantile plots, discrete
  covariates that take 5 or fewer values are jittered for visibility.
  This may be changed by setting this argument to any other positive
  integer.
  
\item {\tt interactive}: If \texttt{TRUE} (default), users can
  identify individual units by clicking on the graph with the left
  mouse button, and (when applicable) choose subclasses to plot.
  
\item {\tt which.xs}: Specifies particular covariate names in a
  character vector to plot only a subset of the covariates.

\item {\tt subclass}: If \texttt{interactive = FALSE}, users can
  specify which subclass to plot. 

\end{itemize}

\subsubsection{Output Values}

\begin{itemize}
\item Empirical quantile-quantile plot: This graph plots covariate
  values that fall in (approximately) the same quantile of treated and
  control distributions.  Control unit quantile values are plotted on
  the x-axis, and treated unit quantile values are plotted on the
  y-axis.  If values fall below the 45 degree line, control units
  generally take lower values of the covariate.  Data points that fall
  exactly on the 45 degree line indicate that the marginal
  distributions are identical.
  
\item Jitter plots: This graph plots jittered estimated propensity
  scores of treated and control units.  Dark diamonds indicate matched
  units and grey diamonds indicate unmatched or discarded units.  The
  area of the diamond is proportional to the weights. Vertical lines
  are plotted if subclassification is used.

\item Histograms: This graph plots histograms of the estimated propensity 
scores in the original treated and control groups and weighted histograms of the estimated
propensity scores in the matched treated and control groups.  Plots can be compared vertically to
quickly check the balance before and after matching.  

\end{itemize}

\subsection{Plot options for the matchit summary object}
You can also send a matchit summary object to the \texttt{plot()} command, to obtain
a summary of the balance on each covariate before and after matching.  The summary() object
must have been created using the option {\texttt standardize=TRUE}.  

\subsubsection{Syntax}

\begin{verbatim}
> s.out <- summary(object, standardize=TRUE, ...)

> plot(s.out, ...)
\end{verbatim}

\subsubsection{Arguments}

\begin{itemize}
\item {\tt interactive}: If \texttt{TRUE} (default), users can
  identify individual variables by clicking on the graph with the left
  mouse button.
\end{itemize}

\subsubsection{Output Values}

\begin{itemize}
\item Line plot of standardized differences in means before and after matching.  
Numbers plotted are those output by the summary() command in the sum.all and sum.matched
objects.  
\end{itemize}



%%% Local Variables: 
%%% mode: latex
%%% TeX-master: "matchit"
%%% End: 

