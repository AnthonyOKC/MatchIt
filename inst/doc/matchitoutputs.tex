Regardless of the type of matching performed, the \texttt{matchit}
output object contains the following elements:\footnote{When
inapplicable or unnecessary, these elements may equal {\tt NULL}.  For
example, when exact matching, {\tt match.matrix = NULL}.}

\begin{enumerate}
\item \texttt{call} provides the original {\tt matchit()} call.
  
\item \texttt{formula} shows the formula used to specify the model for
  estimating the distance measure.
  
\item \texttt{model} stores the output of the model used to estimate
  the distance measure.  \texttt{summary(m.out\$model)} will give the
  summary of the model where \texttt{m.out} is the output object from
  \texttt{matchit()}.
  
\item \texttt{match.matrix} is an $n_1$ by \texttt{ratio} matrix
  where:
  \begin{itemize}
  \item the row names, which can be obtained through
    \texttt{row.names(match.matrix)}, represent the names of the
    treatment units, which come from the data frame specified in
    \texttt{data} (to learn how to do this, see Section~\ref{rnames}).
  \item each column stores the name(s) of the control unit(s) matched
    to the treatment unit of that row. For example, when the
    \texttt{ratio} input for nearest neighbor or optimal matching is
    specified as 3, the three columns of \texttt{match.matrix}
    represent the three control units matched to one treatment unit).
  \item \texttt{NA} indicates that the treatment unit was not matched.
  \end{itemize}

\item \texttt{discarded} is a vector of length $n$ that displays
  whether the units were ineligible for matching due to common support
  restrictions.  It equals \texttt{TRUE} if unit $i$ was discarded,
  and it is set to \texttt{FALSE} otherwise.
  
\item \texttt{distance} is a vector of length $n$ with the estimated
  distance measure for each unit.
  
\item \texttt{weights} is a vector of length $n$ that provides the
  weights assigned to each unit in the matching process.  Unmatched
  units have weights equal to $0$. Matched treated units have weight
  $1$.  Each matched control unit has weight proportional to the
  number of treatment units to which it was matched, and the sum of
  the control weights is equal to the number of uniquely matched
  control units. See Section~\ref{subsec:weights} for more details.
  
\item \texttt{subclass} contains the subclass index in an ordinal
  scale from 1 to the total number of subclasses as specified in
  \texttt{subclass} (or the total number of subclasses from full or
  exact matching).  Unmatched units have \texttt{NA}.
  
\item \texttt{q.cut} gives the subclass cut-points that classify the
  distance measure.
  
\item \texttt{treat} stores the treatment indicator from \texttt{data}
  (the left-hand side of \texttt{formula}).
 
\item \texttt{X} stores the covariates used for estimating the
  distance measure (the right-hand side of \texttt{formula}).  When
  applicable, \texttt{X} is augmented by covariates contained in
  \texttt{mahvars} and \texttt{exact}. 
\end{enumerate}
