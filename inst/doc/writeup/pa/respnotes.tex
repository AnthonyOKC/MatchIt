reviewer 1:

p.1, bottom: we don't do Bayesian model averaging or Leamer's extreme
bounds analysis because once you preprocess (well), all the models you
estimate will give roughly the same answers.  So the model dependence
that is the uncertainty intended to be captured by BMA is no longer an
issue.  We have no objection to BMA or EBA but for those you need to
specific a particular class of models which you will average over or
calculate the bounds for.  In our approach, preprocessing should
reduce model dependence for any model in a broad range. 

p.2, top.  I think there is only one confusion in the literature
(which i realize after our conversation with Alberto). it comes from
the econometricians assuming as little as possible and statisticians
or applied soc scientists assuming full parametric models.  we can
just clarify this (instead of saying that there is controversy over
how to compute se's) and explain that we are trying to help people who
are ALREADY assuming parametric models by adding preprocessing.

p.2, 2nd para: yes there's a bias-variance tradeoff.  with matching
you discard data that, if you were willing to assume the model was
right, would often produce smaller se's.  that's the intuition the
reviewer wanted. but there are 2 qualifications:  1 is that we aren't
willing to assume the parametric model applies to some areas, and 2 is
that in some cases discarding data can reduce variance.  and most
importantly the 'more data is better' rule only applies when the model
is ``self-efficient'' (Meng) and we rule that out when you don't know
what the model is.

3rd para: are there theorems we don't now reference BUT are relevant
to practical application?  I think we decided this was 'no', but we
should check.  the point we make is that most of the thms are really
irrelevant to practice since they assume things we don't know.  the
``good old MSE'' criteria only make sense if you assume there is no
misspecification. 

bottom, p.2 we have leave fig 1 in, but explain more clearly that it
is a hypothetical example.  i think we can also explain also point in
the blog entry I recently wrote (see at the end of this message; it
hasn't been posted yet) about advantages of other approaches and how
they all often give similar answers when done properly.

last para p.2.  section 6 is already a checklist.

top, p.3. we could define model dependence formally.  in King and
Zeng, we define it as:

``For simplicity in this section, we define \emph{model dependence} at
point $x$ as the difference, or distance, between the predicted
outcome values from any two \emph{plausible} alternative models.  (One
model might be logit and the other probit, or one linear the other
quadratic, etc.)  By ``plausible'' alternative models, we mean models
that fit the data reasonably well and, in particular, they fit about
equally well around either the ``center'' of the data (such as a
multivariate mean or median) or the center of a sufficiently large
cluster of data nearest the counterfactual $x$ of interest. It is easy
to generate model dependence when the model does not even fit the
data, but this is easy to avoid, as many current data analysis
techniques are designed to detect and correct bad fitting models.''

sure, we can change 'the' to 'a'

reviewer 2 ==========================================

1. i think we should be willing to this if the editor would really
like us to, but i think our position should be that it is wrong.
i.e., our whole point is that virtually all regression-type models are
conditional on X, and the se's are computed from that perspective.  the
uncertainty is not different whether X is determined by the world, by
some classical experimental design (randomized blocks), by strata in a
survey, or by matching.  the only difference is that only in the
latter case can we optimize: we choose the _best_ (the optimum) among
all matching solutions.  the only reason for us to consider a
different one is if someone can find a better match.  in that case,
our answers may well change.  but this isn't model uncertainty in the
same way as due to models which CANNOT really be ranked in terms of
which is better (BMA not withstanding).

2. 1-to-n matching is ok with us too.  why did we do 1-to-1 in this
example?  and of course we'll make the data available.

3.  the difference is that in the stat 101 example, the selection is
on Y, whereas in matching if you follow the rules you only select on
X, which can cause no bias.

you can look at Rubin's work as adding observations rather than
deleting some, but you'd still have to ask why he doesn't add them
all.  its the same issue.  why do we delete observations?  because
more observations are only better if the model is self-efficient.
otherwise you just get model dependence and have no real advantage.

4.  same comment as for 3.  cite Imai and Van Dyk.  do any of these
alternative methods move you?

5. so we have to better state the formal but irrelevant thm about
pscores.  anyone have a better way to say this?

6.  we're right about balance tests.  no reason to move here.

reviewer 3 ========================================================

a cranky review.  so we just clarify things a bit more.

authors can keep using their parametric model as they had and they'd
be better off.  of course, we recommend that they also use their
parametric model better, and there's a lot of room for improvement
here.  fair enough, but it doesn't affect us much except for some
wording.

``first to work out the conditions...''  we're right here; the
reviewer found no other citation and neither did we.

I like the Ec Inf section, but we can make it short and make it a FN
perhaps; not sure i think it adds a lot, but maybe we give on things
that aren't central.

'doubly robust'.  i think we need to find a more careful way of making
and appropriately qualifying this claim, but i think we should keep
the phrase.
 
'balance test fallacy':  ok, so we also mention (again!) that we
should verify balance on more than just the first moment.

se's:  right some rare methods don't condition on X.  big deal.  the
clarification i mention above about se's should help here too.

why is this different from 2sls.  ugh.  no reason to change the paper
for this but perhaps we can write something for the reviewer.

exactly what semi-parametric alternatives do social scientists know
well?


====================================== blog entry ==========================

A Unified Theory of Statistical Inference?

If inference is the process of using data we have in order to learn
about data we do not have, it seems obvious that there can never be a
proof that anyone has arrived at the "correct" theory of inference.
After all, the data we have might have nothing to do with the data we
don't have. So all the (fairly religious) attempts at unification --
likelihood, Bayes, Bayes with frequentist checks, bootstrapping, etc.,
etc. -- each contribute a great deal but they are unlikely to
constitute The Answer. The best we can hope for is an agreement, or a
convention, or a set of practices that are consistent across fields.
But getting people to agree on normative principles in this area is
not obviously different from getting them to agree on the normative
principles of political philosophy (or any other normative
principles).

It just doesn't happen, and even if it does it would have merely the
status of a compromise rather than the correct answer, the latter
being impossible.

Yet, there is a unifying principal that would represent progress in
the sense that would advance the field: we will know that something
like unification has occurred when we distribute the same data, and
the same inferential question, to a range of scholars with different
theories of inference, that go by different names, use different
conventions, and are implemented with different software, and yet they
all produce approximately the same empirical answer.

We are not there yet, and there are some killer examples where the
different approaches yield very different conclusions, but there does
appear to be some movement in this direction. The basic unifying idea
I think is that all theories of inference require some assumptions,
but we should never take any theory of inference so seriously that we
don't stop to check the veracity of the assumptions. The key is that
conditioning on a model does not work, since of course all models are
wrong, and some are really bad. What I notice is that most of the
time, you can get roughly the same answers using (1) likelihood or
Bayesian models with careful goodness of fit checks and adjustments to
the model if necessary, (2) various types of robust, semi-parametric,
etc. statistical methods, (3) matching for use as preprocessing data
that is later analyzed or further adjusted by parametric likelihood or
Bayesian methods, (4) Bayesian model averaging, with a large enough
class of models to average over, (5) the related "committee methods'',
(6) mixture of experts models, and (7) some highly flexible functional
forms, like neural network models. Done properly, these will all
usually give similar answers.

This is related to Xiao-Li Meng's self-efficiency result: the rule
that ``more data are better'' only holds under the right model.
Inference can't be completely automated for most quantities, and we
typically can't make inferences without some modeling assumptions, but
the answer won't be right unless the assumptions are correct, and we
can't ever know that the assumptions are right. That means that any
approach has to come to terms with the concept that some of the data
might not be right for the given model, or the model might be wrong
for the observed data. Each of the approaches above has an extra
component to try to get around the problem of incorrect models. This
isn't a unification of statistical procedure, or a single unified
theory of inference, but it may be leading to a unification of
results of many diverse procedures, as we take the intuition from each
area and apply it across them all.

